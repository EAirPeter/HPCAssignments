\documentclass{article}

\usepackage[utf8]{inputenc}
\usepackage{amssymb}
\usepackage{amsmath}
\usepackage{bbm}
\usepackage{mathtools}
\usepackage{diagbox}
\usepackage{fullpage}
\usepackage{hyperref}
\usepackage{booktabs}
\usepackage{algorithm}
\usepackage{algpseudocode}
\usepackage{listings}
\usepackage{rotating}
\usepackage{fontspec}
\usepackage{filecontents}
\usepackage{pgfplots}

\setmonofont{Consolas}

\title{Report for HPC Homework 5}
\author{Zhen Li}
\date{\today}

\begin{filecontents*}{weak.csv}
NP,BL,NB
1,0.099529,0.127009
4,0.173452,0.143530
16,0.228434,0.186785
64,1.785567,1.441090
256,3.837413,2.438356
\end{filecontents*}

\begin{filecontents*}{strong.csv}
NP,BL,NB
1,1,1
4,2.283515958,2.237754604
16,8.722399017,7.767556228
64,29.57530104,34.91080837
256,107.5529521,106.3285942
\end{filecontents*}

\begin{document}

\maketitle

\begin{enumerate}
  \item MPI-parallel two-dimensional Jacobi smoother
  \begin{itemize}
    \item Every points reside in the $N_l \times N_l$ matrix should be updated
    by current process.

    \item The leftmost column, rightmost column, topmost row and the
    bottommost row should be communicated to the adjacent left, right, up,
    down process.

    \item See \url{jacobi.cpp} for blocking implementation and
    \url{jacobi-nb.cpp} for non-blocking implementation.

    \item See \url{jobs/**/*.out} for results generated on Prince cluster
    where \lstinline{bl} stands for blocking version, \lstinline{nb} stands
    for non-blocking version, \lstinline{n} stands for $N$, \lstinline{ln}
    stands for $N_l$ and \lstinline{np} stands for number of processes in the
    file names.

    \item For convenience, results are shown in the following table:
    \begin{center}
      \begin{tabular}{cccccc}
        \toprule
        $N$ & $N_l$ & \#Iteration & \#Process & Time (blocking) &
          Time (non-blocking) \\
        \midrule
        100  & 100 & 10000 & 1   & 0.099529 s & 0.127009 s \\
        200  & 100 & 10000 & 4   & 0.173452 s & 0.143530 s \\
        400  & 100 & 10000 & 16  & 0.228434 s & 0.186785 s \\
        800  & 100 & 10000 & 64  & 1.785567 s & 1.441090 s \\
        1600 & 100 & 10000 & 256 & 3.837413 s & 2.438356 s \\
        \midrule
        25600 & 25600 & 100 & 1   & 166.562417 s & 163.037036 s \\
        25600 & 12800 & 100 & 4   & 72.941210  s & 72.857424  s \\
        25600 & 6400  & 100 & 16  & 19.095941  s & 20.989489  s \\
        25600 & 3200  & 100 & 64  & 5.631808   s & 4.670102   s \\
        25600 & 1600  & 100 & 256 & 1.548655   s & 1.533332   s \\
        \bottomrule
      \end{tabular}
    \end{center}
    where the upper half is for weak scalability comparison and the lower half
    is for strong scalability comparison.

    \item For weak scalability, the timings are plotted below. As you may see,
    this algorithm is not weakly scalable since the running time significantly
    increases as $N$ and \#Process grows.
    \begin{center} \begin{tikzpicture}
      \begin{axis} [
        xlabel = {\#Process},
        ylabel = {Time (sec)},
        scale only axis,
        axis x line* = bottom,
        axis y line* = left,
        legend style = {
          at = {(0.1, 0.9)},
          anchor=north west,
        },
      ]
        \addplot table [x=NP, y=BL, col sep=comma]{weak.csv};
        \addplot table [x=NP, y=NB, col sep=comma]{weak.csv};
        \legend{Blocking, Non-blocking}
      \end{axis}
    \end{tikzpicture} \end{center}

    \item For strong scalability, the speedups are plotted below. It some how
    resembles linear speedup, but it is still not ideal.
    \begin{center} \begin{tikzpicture}
      \begin{axis} [
        xlabel = {\#Process},
        ylabel = {Speedup},
        scale only axis,
        axis x line* = bottom,
        axis y line* = left,
        legend style = {
          at = {(0.1, 0.9)},
          anchor=north west,
        },
      ]
        \addplot table [x=NP, y=BL, col sep=comma]{strong.csv};
        \addplot table [x=NP, y=NB, col sep=comma]{strong.csv};
        \addplot table [x=NP, y=NP, col sep=comma]{strong.csv};
        \legend{Blocking, Non-blocking, Ideal}
      \end{axis}
    \end{tikzpicture} \end{center}

    \item For comparison between blocking and non-blocking version, actually
    their timings do not differ much but the non-blocking version still
    outperforms the blocking one when \#Process and $N$ both getting bigger.

  \end{itemize}

\end{enumerate}

\end{document}
